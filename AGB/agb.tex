\documentclass[a4paper, 12pt]{scrartcl}

\usepackage[left=2.5cm,top=3cm, bottom=3cm, right=2cm]{geometry}

\usepackage{secdot} % Dots in Section Numbers
\usepackage[utf8]{inputenc}
\usepackage[T1]{fontenc}
\usepackage[ngerman]{babel}
%\usepackage{amsthm}
%\usepackage{amsmath}
%\usepackage{amssymb}
\usepackage{tabularx} % linewidth tables
%\usepackage{booktabs} % for top- mid- and bottomrule
%\usepackage[european]{circuitikz}
%\usepackage{tikz}
%\usepackage{graphicx} % to include Pictures
%\usepackage{float}
\usepackage{todo}
\usepackage{pdfpages}
\usepackage{csquotes}
%\newcolumntype{R}{>{\raggedleft\arraybackslash}X}

%\usepackage{siunitx} 
%\usepackage{pgfplots}
%\usepackage{pgfplotstable}
%\usepackage{csvsimple}
%\usepackage{url}
%\usepackage{hyperref}

\usepackage{fancyhdr}
\pagestyle{fancy}
\fancyhf{}
\chead{Von Erkenntnistheorie bis Moralphilosophie}
\rhead{\thepage}
%\cfoot{\thepage}
\newcolumntype{C}{>{\centering\arraybackslash}X}
\setlength\extrarowheight{3pt}


%\pgfplotsset{compat=newest} % Allows to place the legend below plot
%\usepgfplotslibrary{units} % Allows to enter the units nicely

\setlength{\parindent}{0pt}
\renewcommand{\labelenumi}{(\arabic{enumi})}
\renewcommand{\labelenumii}{(\arabic{enumii})}
\renewcommand{\labelenumiii}{(\arabic{enumiii})}

%\definecolor{nicegreen}{rgb}{0.09, 0.45, 0.27}

\begin{document}
    \begin{titlepage}
        \begin{center}
            
	    \large {AGB zum Seminar \enquote{Von Erkenntnistheorie bis Moralphilosophie}} \\
            Winter 2019 \\
            \vspace{3cm}
            \textbf{ \Huge Allgemeine Gesch\"aftsbedingungen} \\
            \vspace{0.5cm}
            \large zu \\
            \vspace{1.5cm}
            \textbf{ \Large Von Erkenntnistheorie bis Moralphilosophie} \\
            \large Ein Seminar des Instituts f\"ur philosophische Perspektivenvielfalt

            \vspace{3cm}

            \begin{tabularx}{\textwidth}{CC}
            \textbf{Martin Sonnenberger} & \textbf{Paul Sladen} \\
            Leitender Veranstalter \& Referent & Technik \& Onlinepr\"asenz\\
            \end{tabularx}
           
            \vfill
            
            \begin{tabularx}{\textwidth}{XX}
                Durchführung: & 06.12.2019-08.21.2019
            \end{tabularx}
         \end{center}
    \end{titlepage}

    \section{Geltungsbereich}
    \begin{enumerate}
	    \item Unsere Teilnahme- und allgemeinen Geschäftsbedingungen regeln das Vertragsverhältnis zwischen den Teilnehmern der jeweiligen Veranstaltung und des Instituts f\"ur Philosophische Perspektivenvielfalt (nachfolgend IfPPV) als Veranstalter. 
	    
	    \item Abweichende allgemeine Geschäftsbedingungen der Teilnehmer bzw. der Unternehmen, denen sie angehören, haben keine Gültigkeit.
    \end{enumerate}

    \section{Anmeldung}
    \begin{enumerate}
        \item Anmeldungen sind verbindlich.
        Mit der Zusendung der Anmeldebestätigung und der Rechnung gilt der Vertrag über die Seminarteilnahme als abgeschlossen.
        Geht Ihnen die Anmeldebestätigung nicht oder verzögert zu, so gilt der Vertrag als geschlossen, wenn das IfPPV nicht innerhalb einer Frist von 14 Tagen die Ablehnung erklärt.

        \item Die verbindliche Anmeldung zu den Veranstaltungen kann schriftlich mit den für Sie vorbereiteten Anmeldeformularen per E-Mail oder Fax getätigt werden.
        Falls Sie die von uns vorbereiteten Formulare nicht verwenden, geben Sie bitte den Namen des Teilnehmenden und die vollständige Firmenanschrift bzw. Rechnungsanschrift mit Telefon- und Faxnummer sowie E-Mail-Adresse an.

        \item Ihre Anmeldung wird in der Reihenfolge des Eingangs von gebucht.
        Im Falle der Überbuchung wird der Anmeldende unverzüglich informiert; ein Vertrag kommt in diesem Fall nicht zustande.
        Das IfPPV beh\"alt sich vor, eine Veranstaltungsbuchung ohne Angabe von Gr\"unden abzulehnen.

        \item Es gelten die zum Buchungsdatum im Anmeldeformular angegebenen Preise zzgl. MwSt.
    \end{enumerate}

    \section{Teilnahme}
    \begin{enumerate}
        \item Die Teilnahme am Seminar ist nur in Verbindung mit der Teilnahme an der Feldstudie \enquote{Alternativen des Justizvollzugs Nr. 3} des Ministeriums f\"ur Justiz m\"oglich.
        
        \item Die Allgemeinen Gesch\"aftsbedingungen des Ministeriums f\"ur Justiz zur Feldstudie \enquote{Alternativen des Justizvollzugs Nr. 3} finden Sie an diese AGB angeh\"angt.
        
        \item Die Zustimmung beider AGB findet gemeinsam statt und tritt in Wirksamkeit mit Ihrer Unterschrift in der Einverst\"andniserkl\"arung am Ende dieses Dokuments.

	    \item Teilnahmeberechtigt sind alle Personen \"alter als 16 Jahre mit Wohnsitz innerhalb der Europ\"aischen Union.
    \end{enumerate}

    \section{Haftung}
    \begin{enumerate}
	    \item Das IfPPV haftet nicht für Sch\"aden des Vertragspartners, insbesondere f\"ur solche, die durch Unf\"alle in den Seminar- oder Lehrgangsräumen entstehen.
	    
	    \item Ferner haftet das IfPPV nicht für Sch\"aden, die durch den Verlust oder Diebstahl von in die Seminar- oder Lehrgangsr\"aume mitgebrachten Sachen oder Wertgegenst\"anden entstehen.
	    
        \item Bei leichter Fahrl\"assigkeit von gesetzlichen Vertretern und leitendem Personal von IfPPV oder jeglicher Fahrl\"assigkeit anderer Erf\"ullungsgehilfen haftet IfPPV nur f\"ur den typischerweise vorhersehbaren Schaden.
        Die Haftungsbeschr\"ankung gilt nicht für Personensch\"aden.

	    \item F\"ur die inhaltliche Richtigkeit des Seminar- und Lehrgangsmaterials inklusive m\"undlich erteilten Ausk\"unften von Lehrpersonen, IfPPV-Personal oder anderer Erf\"ullungsgehilfen kann trotz aller Sorgfalt bei der Bearbeitung und Qualit\"atskontrolle keine Gew\"ahr und regelm\"a{\ss}ig keine Haftung \"ubernommen werden.
	    
        \item Die Seminare des IfPPV stellen keine medizinische oder psychotherapeutische Behandlung dar. 
        Es ist nicht geeignet, um psychische St\"orungen aufzuarbeiten. 
        Die Teilnahme an den Seminaren erfolgt freiwillig und eigenverantwortlich.

        \item Anspr\"uche des Teilnehmers auf Schadensersatz gleich welchen Rechtsgrundes sind ausgeschlossen.
        Hiervon ausgenommen sind Schadensersatzanspr\"uche des Teilnehmers aus der Verletzung des Lebens, des K\"orpers, der Gesundheit.

        \item Das IfPPV w\"ahlt für die Seminare in den jeweiligen Bereichen qualifizierte Referenten aus.
        F\"ur die Korrektheit, Aktualit\"at und Vollst\"andigkeit der Seminarinhalte, der Seminarunterlagen sowie die Erreichung des jeweils vom Teilnehmer angestrebten Ziels \"ubernimmt das IfPPV keine Haftung.
        Ebenso nicht für etwaige Folgesch\"aden, welche aus fehlerhaften und/oder unvollst\"andigen Seminarinhalten entstehen sollten.
        Im \"Ubrigen ist die Haftung des IfPPV auf Vorsatz, grobe Fahrl\"assigkeit und die Verletzung vertragswesentlicher Pflichten beschr\"ankt, wobei es sich um typische, bei einer Seminarveranstaltung vorhersehbare Sch\"aden handeln muss.
    \end{enumerate}
    
    \section{Datenschutz}
    \begin{enumerate}
        \item Alle Daten der Teilnehmenden werden vertraulich behandelt und nur im Einklang mit den datenschutzrechtlichen Bestimmungen der Bundesrepublik Deutschland und der DSGVO genutzt.

        \item Daten der Teilnehmenden werden zum Zwecke der Abrechnung und Leistungserbringung des IfPPV gespeichert und nur soweit zu diesem Zwecke notwendig an daf\"ur beauftragte Dienstleister weitergegeben.
    \end{enumerate}
    
    \section{Absage von Veranstaltungen}
    \begin{enumerate}
        \item Das IfPPV beh\"alt sich vor, die Veranstaltung wegen zu geringer Teilnehmerzahl oder aus sonstigen wichtigen, von uns nicht zu vertretenden Gr\"unden (z. B. h\"ohere Gewalt, pl\"otzliche Erkrankung des Referenten) abzusagen.
        Bereits von Ihnen entrichtete Teilnahmegeb\"uhren werden nicht zur\"uckerstattet.
        Dar\"uber hinaus ergeben sich keine weiteren Anspr\"uche gegen das IfPPV.
    \end{enumerate}

    \section{\"Anderungen an dieser Vereinbarung}
    \begin{enumerate}
        \item Wir behalten uns das Recht vor, in angemessener Weise \"Anderungen an dieser Vereinbarung vorzunehmen, um beispielsweise \"Anderungen unseres Seminars zu ber\"ucksichtigen oder aus rechtlichen, regulatorischen oder Sicherheitsgr\"unden.
        Das IfPPV wird \"uber alle wesentlichen \"Anderungen an dieser Vereinbarung rechtzeitig informieren und Ihnen die Gelegenheit geben, diese zu \"uberpr\"ufen.
        \"Anderungen, die sich auf neu verf\"ugbare Vorlesungen des Seminars beziehen oder \"Anderungen aus rechtlichen Gr\"unden k\"onnen jedoch sofort wirksam werden.
        \"Anderungen sind ausschließlich f\"ur die Zukunft wirksam.
        Wenn Sie den ge\"anderten AGB nicht zustimmen, sollten Sie das Seminar verlassen.
    \end{enumerate}

    \section{Anwendbares Recht}
    \begin{enumerate}
        \item F\"ur s\"amtliche Rechtsbeziehungen der Parteien gilt das Recht der Bundesrepublik Deutschland.

        \item Abweichende Allgemeine Gesch\"aftsbedingungen von Vertragspartnern, Teilnehmenden bzw. Organisationen finden keine Anwendung, soweit sie nicht von IfPPV Managementberatern schriftlich anerkannt werden.

        \item Die Allgemeinen Gesch\"aftsbedingungen unterliegen dem Recht der Bundesrepublik Deutschland unter Ausschluss des UN Kaufrechts.
        Sollten einzelne Bestimmungen dieser AGB ganz oder teilweise unwirksam oder erg\"anzungsbed\"urftig sein oder werden, so bleibt die Wirksamkeit der \"ubrigen Bestimmungen unber\"uhrt.
        Die Vertragspartner werden anstelle der unwirksamen oder erg\"anzungsbed\"urftigen Bestimmungen eine neue Regelung vereinbaren, die dem gewollten wirtschaftlichen Zweck am n\"achsten kommt.

        \item IfPPV-Managementberater behalten es sich vor, diese AGB zu \"andern und es gilt die jeweilig letzte Version.

        \item Ist der Kunde Kaufmann oder juristische Person des \"offentlichen Rechts, ist ausschließlicher Gerichtsstand f\"ur alle Streitigkeiten aus diesem Vertrag der Gesch\"aftssitz der BASA.
    \end{enumerate}

    \section{Urheberrecht}
    \begin{enumerate}
        \item Alle im Rahmen der Veranstaltungen ausgegebenen Unterlagen sind urheberrechtlich geschützt.
        Unsere Seminarinhalte dienen ausschließlich dem persönlichen Gebrauch der Teilnehmer.
        Vervielfältigungen und anderweitige Nutzung sind schriftlich durch das IfPPV zu genehmigen.
        
        \item Zudem behalten wir uns daher vor, Wettbewerber bzw. deren Mitarbeiter von einer Teilnahme auszuschließen.
        
        \item Die Meinung der Referenten spiegelt nicht die Meinung des IfPPV wieder.
        
        \item Die Urheberrechte aller von Teilnehmern erstellten G\"uter liegen bei dem Veranstalter, IfPPV.
    \end{enumerate}

    \section{Schlussbestimmung}
    \begin{enumerate}
	\item Der Vertrag unterliegt dem deutschen Recht. Der Gerichtsstand ist Frankfurt.
    \end{enumerate}

    \newpage
    \includepdf[pages=-]{study.pdf}

    \newpage
    \includepdf[pages=-]{einverstaendnis.pdf}
\end{document}
